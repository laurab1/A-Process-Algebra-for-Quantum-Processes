\documentclass{beamer}
\usepackage[utf8]{inputenc}
\usepackage{listings} % Include the listings-package
\usepackage{graphicx}
\usepackage{booktabs}
\usepackage{xcolor}
\usepackage{wrapfig}
\usepackage{mathtools}
\usepackage{spverbatim}

\DeclarePairedDelimiter\abs{\lvert}{\rvert}%
\DeclarePairedDelimiter\bra{\langle}{\rvert}
\DeclarePairedDelimiter\ket{\lvert}{\rangle}
\DeclarePairedDelimiterX\braket[2]{\langle}{\rangle}{#1 \delimsize\vert #2}

\def\odiv{\, {\ominus\hspace{-8.5pt} \div} \,}
\definecolor{unipi}{HTML}{294488}
\usetheme{Boadilla}
\setbeamercolor{normal text}{fg=unipi}

\title{A process algebra for quantum processes}
\author{Laura Bussi}
\titlegraphic{\includegraphics[width=100pt,height=57pt]{marchio_unipi_pant541.png}}
\date{14/06/2019}
\institute[Università di Pisa]

\begin{document}

\frame{\titlepage}

\begin{frame}
\frametitle{Outline}
\tableofcontents
\end{frame}

\section{Quantum information and security}
 
\begin{frame}
\frametitle{Quantum information}
Quantum information is the information of the state of a quantum system. As classical information is represented by using bits, quantum information is epitomized by qubits. \\
Bits and qubits fundamentally differ:
\begin{itemize}
	\item a classical bit can be only either 0 or 1;
	\item a qubit can be in a state 0 or 1, or in a \emph{coherent superposition} of both;
	\item coherence holds until the qubit is not measured: measurment of a qubit cause 
	the immediate collapse of the qubit in one between the two states with a 
	certain probability.
\end{itemize}
\end{frame}

\begin{frame}
\frametitle{The Dirac notation}
Standard representation of qubits relies on the Dirac notation, also known as bra-ket notation:
\begin{itemize}
	\item as we can choose an orthonormal basis of the linear vector space, we can write 
	a coherent superposition as $\alpha \ket{0} + \beta \ket{1}$.
	\item $\ket{0}$ and $\ket{1}$ are, respectevly, the two orthonormal vectors of 
	the basis: 
	\begin{align*}
	\ket{0} = \begin{bmatrix}
           1 \\
           0 \\
         \end{bmatrix}
         & &
    \ket{1} = \begin{bmatrix}
    	   0 \\
    	   1 \\
    	 \end{bmatrix}
    \end{align*}
    \item $\alpha$ and $\beta$ can be in general complex numbers. Given the above state, 
    we have that a measurment of that state will result in observing 0 with probability 
    $\abs{\alpha^2}$ and 1 with probability $\abs{\beta^2}$. Obviously, it must hold 
    $\abs{\alpha^2} + \abs{\beta^2} = 1$.
\end{itemize}
\end{frame}

\begin{frame}
\frametitle{Pure and mixed states}
As said, a qubit in a \emph{pure state} is fully represented by a linear combination $\alpha \ket{0} + \beta \ket{1}$, i.e. a single ket vector in a Hilbert space. \\
However, due to its nature, quantum states are difficult to isolate and might be entangled with the enviroment. Then we have to consider the so called \emph{mixed state}, i.e. a collection of pure states $\ket{\psi_i}$ each associated with a probability $p_i$. \\
Again, it must hold $0 \leq p_i \leq 1$ and $\sum_i p_i = 1$. \\
In order to handle quantum states which include mixed states, we introduce the \emph{density operator} (or \emph{density matrix}, denoted as $\rho$) representation:
	\begin{align*}
		\rho &= \sum_i p_i \ket{\psi_i}\bra{\psi_i}
	\end{align*}
\end{frame}

\begin{frame}
\frametitle{Operations on qubits and measurements}
\emph{Quantum logic gates} are used to implement operations on qubits. Operations are performed over qubits in a pure state and include creation and removal of superposition (\emph{Hadamard operator}) and negation of a qubit w.r.t. the state of another qubit (\emph{Controlled-NOT}). The latter is a primary component in building a quantum computer.
	\begin{align*}
		H &= \frac{1}{\sqrt{2}}\begin{bmatrix}
           1 & 1\\
           1 & -1\\
         \end{bmatrix}
         & & &
        CNOT &= \begin{bmatrix}
           1 & 0 & 0 & 0\\
           0 & 1 & 0 & 0\\
           0 & 0 & 0 & 1\\
           0 & 0 & 1 & 0\\
         \end{bmatrix}
	\end{align*}
Measurment of a qubit is an irreversible operation. In particular, a measurment of a qubit whose state is entangled with another system will transform the pure state of the qubit in a mixed state. 
\end{frame}

\begin{frame}
\frametitle{What about security?}
Quantum computing is both a possibility and a threat:
\begin{itemize}
	\item fundamental theorems of quantum information states that qubits cannot be copied
	nor broadcasted;
	\item the BB84 key distribution protocol, that we're going to se later on,
	exploits the fact that qubits cannot be measured without changing their
	state.
	\item On the other hand, the larger amount of information held by a single qubit
	compared to a classical bit causes a sensible speedup in solving traditionally "hard"
	problems, on which the security of many cryptographic protocols relies on.
	\item Thus, dissemination of quantum computers may result in having no longer secure
	 protocols.
\end{itemize}
\end{frame}

\section{A process algebra for quantum systems: qCCS}

\begin{frame}
\frametitle{Towards a process algebra for quantum systems}
As they involve quantum information, designing quantum protocols is surely error prone, even more than designing classical protocols. Having a formal language that can be used to design and verify quantum protocols plays a fundamental role in quantum security. \\
Process algebras are expressive models for concurrent computation in classical computing. In the following we'll see an extension of CCS for handling quantum processes, called qCCS.
\end{frame}

\subsection{A probabilistic model for qCCS}

\begin{frame}
\frametitle{Probabilistic Labelled Transition Systems}
As seen, quantum measurments give rise to probability distributions, thus it is quite natural to model the evolution of a quantum process by using pLTSs. \\
We consider then a probabilty distribution over a set $S$:
	\begin{align*}
		\Delta : S \rightarrow [0,1] & & s.t. \ & \sum_s \Delta(s) = 1
	\end{align*}
The set $\lceil \Delta \rceil = \{ s \mid \Delta(s) > 0 \}$ is called the support of 
$\Delta$. We call $\overline{s}$ a \emph{point distribution}, i.e. a distribution s.t. $\lceil \overline{s} \rceil = \{ s \}$. \\
We denote as $Dist(S)$ the set of finite support distributions over S. \\
Then, a pLTS is a triple $\langle S, Act_\tau, \rightarrow \rangle$ where $S$ is a set of states, $Act_\tau$ is a set of labels $Act \cup \{ \tau \}$ and $\rightarrow \subseteq S \times Act_\tau \times Dist(S)$.
\end{frame}

\begin{frame}
\frametitle{Lifting the transition relation}
Since we want to allow distributions over states to perform an action, we have to lift the relations $\xrightarrow\alpha$. \\
Let $\mathcal{R} \subseteq S \times Dist(S)$ be a relation from states to distributions in a pLTS. Then $\mathcal{R}^\circ \subseteq Dist(S) \times Dist(S)$ is the smallest relation satisfying the following rules:
	\begin{itemize}
		\item $s \ \mathcal{R} \ \Delta \implies \overline{s} \ \mathcal{R}^\circ \ \Delta$
		\item $\forall i \in I. \ \Theta_i \ \mathcal{R} \ \Delta_i \implies (\sum_{i \in I} p_i \Theta_i) \ \mathcal{R}^\circ \ (\sum_{i \in I} p_i \Delta_i)$, for $p_i \in [0,1]$, $\sum_{i \in I} p_i = 1$, where $I$ is a countable index set.
	\end{itemize}
For a weak transition $\xRightarrow{\overline{\alpha}}$ we write $\Delta \xRightarrow{\overline{\alpha}} \Theta$ whenever $\Delta \xRightarrow{\overline{\tau}} \xrightarrow\alpha \xRightarrow{\overline{\tau}} \Theta$.
\end{frame}

\subsection{qCCS: syntax}

\begin{frame}
\frametitle{Syntax of qCCS}
Three types of data are considered:
	\begin{itemize}
		\item \emph{bool} for booleans;
		\item \emph{real} for real numbers;
		\item \emph{Qbt} for qubits.
	\end{itemize}
As a consequence, we have two infinite sets of variables:
	\begin{itemize}
		\item \emph{cVar} for classic variables;
		\item \emph{qVar} for quantum variables.
	\end{itemize}
We also assume a set \emph{Exp} of expressions over real numbers and a set \emph{BExp} of expressions over booleans.
\end{frame}

\begin{frame}
\frametitle{Syntax of qCCS}
Of course, we need two types of channels:
	\begin{itemize}
		\item \emph{cChan} for classical channels, denoted as \emph{c, d}...
		\item \emph{qChan} for quantum channels, denoted as \emph{\underline{c}, \underline{d}}... 
	\end{itemize}
A relabelling function \emph{f} is a map on $cChan \cup qChan$ such that $f(cChan) \subseteq cChan$ and $f(qChan) \subseteq qChan$. \\
We write $\widetilde{q}$ for a sequence of distinct variables $q_1,...,q_n$.
The terms in qCCS are generated by the following grammar: \\
	\[ P, Q ::= nil \mid \tau.P \mid c?x.P \mid c!x.P \mid \underline{c}?x.P \mid \underline{c}!x.P \mid \mathcal{E}[\widetilde{q}].P \]
	\[ \mid M[\widetilde{q};x].P \mid P + Q \mid P \lvert \rvert Q \mid P[f] \mid P \setminus L \mid if \ \bold{b} \ then \ P \mid A(\widetilde{q}; \widetilde{x}) \]
\end{frame}

\begin{frame}
\frametitle{Free quantum variables}
Free quantum variables are defined inductively:
	\begin{align*}
		qv(nil) = \emptyset && qv(\tau.P) = qv(P)\\
		qv(c?x.P) = qv(P) && qv(c!e.P) = qv(P)\\
		qv(\underline{c}?q.P) = qv(P) \setminus {q} && 
		qv(\underline{c}!q.P) = qv(P) \cup {q}\\
		qv(\mathcal{E}[\widetilde{q}].P) = qv(P) \cup \widetilde{q} &&
		qv(M[\widetilde{q};x].P) = qv(P) \cup \widetilde{q}\\
		qv(P + Q) = qv(P) \cup qv(Q) && qv(P \lvert \rvert Q) = qv(P) \cup qv(Q)\\
		qv(P[f]) = qv(P) && qv(P \setminus L) = qv(P)\\
		qv(if \ \bold{b} \ then \ P) = qv(P) &&
		qv(A(\widetilde{q};\widetilde{x})) = \widetilde{q}		
	\end{align*}
For a process to be legal we require that:
	\begin{itemize}
		\item $q \not \in qv(P)$ in $\underline{c}!q.P$
		\item $qv(P) \cap qv(Q) = \emptyset$ in $(P \lvert \rvert Q)$
		\item Each constant $A(\widetilde{p};\widetilde{x})$ has a defining equation
		$A(\widetilde{p};\widetilde{x}) = P$, where $P$ is a term with
		$qv(P) \subseteq \widetilde{p} = \emptyset$ and $fv(P) \subseteq \widetilde{x}$
	\end{itemize}
\end{frame}

\subsection{qCCS: semantics}

\begin{frame}
\frametitle{Informal semantics of qCCS}
Since qCCS is an extension of value-passing CCS, classical operators behaves as expected. \\
As for the process $\underline{c}?q.P$, it receives a quantum datum over the quantum channel $\underline{c}$ and evolves into $P$, while $\underline{c}!q.P$ sends a quantum datum over the quantum channel $\underline{c}$ and evolves into $P$. \\
$\mathcal{E}$ represents a trace-preserving super-operator, i.e. a local quantum operation acting on a mixed state, applied to $\widetilde{q}$. The set of all the trace-preserving super-operators is denoted as $TSO_H$, for $H$ being a (finite dimensional) Hilbert 
space. \\
The process $M[\overline{q};x].P$ measures the states of the sequence $\overline{q}$ and stores the result into the classical variable $x$. \\
\end{frame}
 
\begin{frame}
\frametitle{Operational semantics of qCCS}
\center \includegraphics[width=300pt,height=200pt]{Op_semantics.png}
\end{frame}

\subsection{qCCS: weak bisimulation}

\begin{frame}
\frametitle{Equivalence for quantum processes}
Given a process $P$ and a density operator $\rho$, a configuration is a pair $(P, \rho)$. \\
In order to define equivalence between two processes, we define three criteria:
	\begin{itemize}
		\item \emph{Barb-preservation} holds when two processes have the same probability
		to send out values on classical channels.
		\item \emph{Reduction-closure} ensures that non-deterministic choices are in some
		sense preserved.
		\item \emph{Compositionality} holds if $C \ \mathcal{R} \ D \implies 
		(C \lvert \rvert R) \ \mathcal{R} \ (D \lvert \rvert R)$, with 
		$qv(R) \cap (qv(C) \cup qv(D)) = \emptyset$ and $\mathcal{R}$ closed under
		trace-preserving operators.
	\end{itemize}
A reduction barbed congruence ($\approx_r$) is the largest relation over configurations which is barb-preserving, reduction-closed and compositional. Furthermore, we have that $C \approx_r D \implies qv(C) = qv(D)$ and $env(C) = env(D)$.
\end{frame}

\begin{frame}
\frametitle{Open bisimulation}
Checking for equivalence using the above definition is quite difficult. \\
We define a relation (called \emph{open bisimulation}) which can be used to characterize reduction barbed congruence. \\
A relation $\mathcal{R} \subseteq Con \times Con$ is an open bisimulation if
	\begin{itemize}
		\item $C \mathcal{R} D \implies qv(C) = qv(D), env(C) = env(D)$ and, for any
		trace-preserving operator $\mathcal{E}$ we have that whenever $\mathcal{E}(C)
		\xrightarrow\alpha \Delta$ there is some $\Theta$ with $\mathcal{E}(D)
		\xRightarrow{\overline{\alpha}} \Theta$ and $\Delta \ \mathcal{R}^\circ \ \Theta$.
		\item Both $\mathcal{R}$ and $\mathcal{R}^{-1}$ are open bisimulation.
	\end{itemize}
We denote as $\approx_o$ the largest open bisimulation. Two quantum processes $P$ and $Q$ are bisimilar ($P \approx_o Q$) if, for any quantum state $\rho$ and any indexed set of classical values $\widetilde{v}$ we have 
$\langle P\{\widetilde{v} / \widetilde{x}\}, \rho \rangle \approx_o 
\langle Q\{\widetilde{v} / \widetilde{x}\}, \rho \rangle$.
\end{frame}

\begin{frame}
\frametitle{Ground bisimulation}
In order to simplify the proof of bisimilarity, we can separate the issues to consider 
$TSOs$ and transitions. We define then a new relation. \\
A relation $\mathcal{R} \subseteq Con \times Con$ is a \emph{ground bisimulation} if
	\begin{itemize}
		\item $C \ \mathcal{R} \ D \implies qv(C) = qv(R), env(C) = env(R)$ and whenever 
		$C \xrightarrow\alpha \Delta$ there is some distrubution $\Theta$ with
		$D \xRightarrow{\overline{\alpha}} \Theta$ and $\Delta \ \mathcal{R}^\circ \ 
		\Theta$.
		\item Both $\mathcal{R}$ and $\mathcal{R}^{-1}$ are ground bisimulation.
	\end{itemize}
If $\mathcal{R}$ is a ground bisimulation and it is closed under super-operators, then $\mathcal{R}$ is an open bisimulation. The relation $\approx_o$ is the largest ground bisimulation which is closed under $TSOs$. \\
Furhtermore, $\approx_o$ is preserved by all the constructor of qCCS except for summation.
\end{frame}

\section{Protocols verification: BB84}

\begin{frame}
\frametitle{BB84: informal description}
BB84 was developed by Bennet and Brassard in 1984. It provides a provable secure way to create and share a private key between two parts (Alice and Bob in the following). \\
The basic protocol is defined as follows:
	\begin{itemize}
		\item[1] Alice randomly generates two strings of bits $\widetilde{B}_a$ and 
		$\widetilde{K}_a$, each of size $n$.
		\item[2] Alice prepares a string $\widetilde{q}$ of size $n$ and sends it to Bob.
		\item[3] Bob randomly generates a string of bits $\widetilde{B}_b$ of size $n$.
		\item[4] Bob measures $\widetilde{q}$ according to the basis determined by the
		string of bits he generated. Let the result be $\widetilde{K}_b$.
		\item[5] Bob sends his measurement choice $\widetilde{B}_b$ to Alice, while Alice
		sends $\widetilde{B}_a$ to Bob.
		\item[6] Alice and Bob determine which bits in $\widetilde{B}_a$ and 
		$\widetilde{B}_b$ are equal and discard the bits in $\widetilde{K}_a$ and
		$\widetilde{K}_b$ where the corresponding bits of $\widetilde{B}_a$ and 
		$\widetilde{B}_b$ differ.
	\end{itemize}
\end{frame}

\begin{frame}
\frametitle{BB84: formalisation}
Using qCCS, we can formalise BB84 as follows:
\[ Alice = Ran[\widetilde{q};\widetilde{B}_a].Ran[\widetilde{q};\widetilde{K}_a].Set_{\widetilde{K}_a}[\widetilde{q}].H_{\widetilde{B}_a}[\widetilde{q}].A2B!\widetilde{q}.WaitA(\widetilde{B}_a,\widetilde{K}_a) \]
\[ Wait(\widetilde{B}_a,\widetilde{K}_a) = a2b?\widetilde{B}_b.a2b!\widetilde{B}_a.key_a!cmp(\widetilde{K}_a,\widetilde{B}_a,\widetilde{B}_b).nil \]

\[ Bob = A2B?\widetilde{q}.Ran[\widetilde{q}';\widetilde{B}_b].M_{\widetilde{B}_b}[\widetilde{q};\widetilde{K}_b].b2a!\widetilde{B}_b.Wait(\widetilde{B}_b,\widetilde{K}_b) \]
\[ Wait(\widetilde{B}_b,\widetilde{K}_b) = a2b?\widetilde{B}_a.key_b!cmp(\widetilde{K}_a,\widetilde{B}_a,\widetilde{B}_b).nil \]

\[ BB84 = (Alice \lvert \rvert Bob) \backslash \{a2b,b2a,A2B\} \]
\end{frame}

\begin{frame}
\frametitle{BB84: correctness}
To show the correctness of BB84, we let:
\[ BB84_{spe} = Ran[\widetilde{q};\widetilde{B}_a].Ran[\widetilde{q};\widetilde{K}_a].Ran[\widetilde{q}';\widetilde{B}_b]. \] 
\[(key_a!cmp(\widetilde{K}_a,\widetilde{B}_a,\widetilde{B}_b).nil \lvert \rvert key_b!cmp(\widetilde{K}_a,\widetilde{B}_a,\widetilde{B}_b).nil)\]
It can be checked from the pLTS that $BB84 \approx_o BB84_{spe}$, here for the case $n = 2$:
\center \includegraphics[width=220pt,height=110pt]{plts.png}
\end{frame}

\begin{frame}
\frametitle{BB84 vs Eve}
To test security of BB84, we can take into account an eavesdropper (named Eve) who tries a typical man-in-the-middle attack.
\[ Alice' = (Alice \lvert \rvert key_a?\widetilde{K}_a'.Pstr_{\widetilde{K}_a'}[\widetilde{q};\widetilde{x}].a2b!\widetilde{x}.a2b!SubStr(\widetilde{K}_a',\widetilde{x}).b2a?\widetilde{K}_b''.\] 
\[(if \ SubStr(\widetilde{K}_a',\widetilde{x}) = \widetilde{K}_b'' \ then \ key_a'!RemStr(\widetilde{K}_a',\widetilde{x}) \ else \ alarm_a!0.nil)) \backslash \{key_a\} \]

\[ Bob' = (Bob \lvert \rvert key_b?\widetilde{K}_b'.a2b?\widetilde{x}.a2b?\widetilde{K}_a''.b2a!SubStr(\widetilde{K}_b',\widetilde{x}). \]
\[ (if \ SubStr(\widetilde{K}_b',\widetilde{x}) = \widetilde{K}_a'' \ then \ key_b'!RemStr(\widetilde{K}_b',\widetilde{x}) \ else \ alarm_b!0.nil)) \backslash \{key_b\} \]

\[ Eve = A2E?\widetilde{q}.Ran[\widetilde{q}'',\widetilde{B}_e].M_b[\widetilde{q};\widetilde{K}_e].Set_{\widetilde{K}_e}[\widetilde{q}].H_{\widetilde{B}_e}[\widetilde{q}].E2B!\widetilde{q}.key_e'!\widetilde{K}_e\]
\end{frame}

\begin{frame}
\frametitle{Testing BB84 in a malicious environment}
Now the system must include Eve (with $f_a (A2B) = A2E$ and $f_b (A2B) = E2B$):
\[ BB84_E = ( Alice'[f_a] \lvert \rvert Eve \lvert \rvert Bob'[f_b] ) \backslash \{a2b, b2a, A2E, E2B \}\]
The test is defined as follows:
\[ TestBB84 = (BB84_E \lvert \rvert key_a'?\widetilde{x}.key_b'?\widetilde{y}.key_e'?\widetilde{z}.\]
\[(if \ \widetilde{x} \not = \widetilde{y} \ then \ fail!0.nil \ else \ key_e!\widetilde{z}.skey!\widetilde{x}.nil)) \backslash \{key_a', key_b', key_e'\}\]
\end{frame}

\begin{frame}
\frametitle{Open bisimulation, again}
Proving the security of BB84 is in general very hard. \\
Again, open bisimulation can help to simplify the test. Indeed we can define a simpler test:
\[ TB = Ran[\widetilde{q};\widetilde{B}_a].Ran[\widetilde{q};\widetilde{K}_a].Ran[\widetilde{q}'';\widetilde{B}_e].Ran'_{\widetilde{B}_a,\widetilde{B}_e,\widetilde{K}_a}[\widetilde{q};\widetilde{K}_e].Ran[\widetilde{q}';\widetilde{B}_b].\]
\[ Ran'_{\widetilde{B}_e,\widetilde{B}_b,\widetilde{K}_e}[\widetilde{q};\widetilde{K}_b].Pstr_{\widetilde{K}_{ab}}[\widetilde{q}_a;\widetilde{x}].(if \ \widetilde{K}_{ab} = \widetilde{K}_{ab} \ then \]
\[ key_e!\widetilde{K}_e.skey!RemStr(\widetilde{K}_{ab},\widetilde{x}) \ else \]
\[ (if \ \widetilde{K}_{ab} \not = \widetilde{K}_{ab} \ then \ (alarm_a!0.nil \lvert \rvert alarm_b!0.nil) \ else \ fail!0.nil)) \]
We have that $TB \approx_o TestBB84$: proving correctness of $TB$ suffices then to prove security of BB84 against Eve's attack.
\end{frame}

\section{Related works}

\begin{frame}
\frametitle{Related works}
QCCS is not the only available process algebra for quantum systems. \\
\smallskip
A first definition was provided by Marie Lalire and Philippe Jorrand in their 2004 paper, presenting QPAlg. \\
QPAlg, as qCCS, is inspired by CCS. It provides rules for applying unitary quantum operators, measurments and the ability to send and receive qubits. Equivalence between QPAlg processes is obtained defining \emph{probabilistic branching bisimilarity}; however, such a bisimilarity is shown to be preserved by all operators except for parallel composition. \\ \smallskip
Another process algebra has been introduced in 2004 by Simon Gay and Rajagopal Nagarajan.
Unlike QPAlg and qCCS, CQP is inspired by $\pi$-Calculus. The operational semantics is defined by using reductions, under the assumption that no external communication involving qubits happen. This limits name mobility to classical channels.
\end{frame}

\section{Conclusions and future works}

\begin{frame}
\frametitle{Conclusions and future works}
Quantum computing is probably going to play a fundamental role in the future of security. Adapting reduction barbed congruence to quantum processes and separating super-operator applications from transitions results in having a quite simple technique to prove behavioural equivalence of processes, which can be used to prove security properties. \\
A simplified version of qCCS supports a semi-automated verification tool, introduced in 2016 by Feng et al. \\
The application of a sequential quantum programming language, as for instance QPL, to verify equations in such a tool is a possible future work. \\
More in general, an interesting evolution could be providing name mobility for quantum processes, taking the existing quantum process algebras as a starting point.
\end{frame}

\section{References}

\begin{frame}
\frametitle{References}
\begin{itemize}
	\item https://en.wikipedia.org/wiki/Quantum$\_$information
	\item Michael A. Nielsen, Isaac L. Chuang - Quantum Computing and Quantum Information
	\item Ittoop V. Puthoor - Theory and Applications of Quantum Process Calculus
	\item Simon Gay, Rajagopal Nagarajan - Communicating Quantum Processes
	\item Marie Lalire, Philippe Jorrand - A Process Algebraic Approach to Concurrent and
	Distributed Quantum Computation: Operational Semantics
	\item Yuxin Deng, Yuan Feng - Open Bisimulation for Quantum Processes
\end{itemize}
\end{frame}

\end{document}